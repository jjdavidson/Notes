\documentclass{article} 

% Document layout
\usepackage[
	margin=0.75in,
	headheight=17pt,
	includehead,
	includefoot,
	heightrounded,
]{geometry}

% Headers and footers
\usepackage{fancyhdr}
\fancyhf{}
\pagestyle{fancy}
\fancyhead[L]{Linear Algebra Done Right}
\fancyhead[R]{\rightmark}
\fancyfoot[L]{Notes by Jonathan Davidson}
\fancyfoot[C]{\thepage}

% Colors
\usepackage[dvipsnames]{xcolor} 

% Tcolorbox
\usepackage{tcolorbox}
\newtcolorbox[auto counter,number within=section]{defn}[1]{
	colback=NavyBlue!5!white,
	colframe=NavyBlue!75!black,
	fonttitle=\bfseries,
    title=Definition~\thetcbcounter. #1
}
\newtcolorbox[auto counter,number within=section]{thm}[1]{
	colback=Mahogany!5!white,
	colframe=Mahogany!75!black,
	fonttitle=\bfseries,
    title=Theorem.~\thetcbcounter: #1
}

% Math symbols and theorem environment 
\usepackage{amsmath,amsthm,amssymb}

% Multiple columns
\usepackage{multicol}

% Adjust enumerated lists
\usepackage[shortlabels]{enumitem} 

% Insert Images
\usepackage{graphicx}
\graphicspath{ {./images/} }

% Custom commands
\newcommand{\real}{\mathbb{R}} 
\newcommand{\complex}{\mathbb{C}} 
\newcommand{\field}{\mathbb{F}} 
\newcommand{\set}[2]{\left\{#1\;\vert\;#2\right\}}

% DOCUMENT
\begin{document}
\section{Vector Spaces}
Linear algebra is the study of linear maps on finite dimensional vector spaces.
In linear algebra, complex numbers are necessary for classifying the basic types
of linear transformations in terms of their eigenvalues. Eigenvalues arise as
solutions to certain polynomial equations.

\begin{defn}{complex numbers}
    A \textbf{complex number} is an ordered pair $(a,b)$ of real numbers.
    Complex numbers are often written as $a+bi$. The set of all complex numbers
    is denoted $\complex$.
    \[
        \complex = \set{a+bi}{a,b\in\real}
    \]
    \textbf{Addition} and \textbf{multiplication} on $ \complex $ are defined by
    \begin{gather*}
        (a+bi) + (c+di) = (a+c) + (b+d)i \\
        (a+bi)(c+di) = (ac-bd) + (ad+bc)i
    \end{gather*}
\end{defn}

\noindent The complex numbers $ a+0i $ can be identified with the real numbers $ \real
$. Using multiplication as defined above, it is easy to see that $ i^2 = -1 $.
The complex numbers inherit almost all the properties from the real numbers.
\begin{description}
    \item[ \textbf{Commutativity}] $ \alpha+\beta = \beta +\alpha $ and $ 
        \alpha\beta = \beta\alpha $ for all $ \alpha,\beta\in\complex $ 
    \item[ \textbf{Associativity}] $ (\alpha+\beta)+\gamma = 
        \alpha+(\beta+\gamma) $ and $ (\alpha\beta)\gamma =
        \alpha(\beta\gamma) $ for all $ \alpha,\beta,\gamma\in\complex $ 
    \item[ \textbf{Identities}] $ \alpha+0 = \alpha $ and $ \alpha 1 = \alpha $
        for all $ \alpha\in\complex $ 
    \item[ \textbf{Additive Inverse}] for every $\alpha\in\complex$, there is a
        unique $ -\alpha\in\complex $ such that $ \alpha +(-\alpha) = 0 $ 
    \item[ \textbf{Multiplicative Inverse}] for every $\alpha\in\complex$, there is a
        unique $ \alpha^{-1}\in\complex $ such that $ \alpha\alpha^{-1} = 1 $ 
    \item[ \textbf{Distributive Property}] $ \gamma(\alpha+\beta) = \gamma\alpha 
        + \gamma\beta$ for all $ \alpha,\beta,\gamma\in\complex $ 
\end{description}

\noindent These properties for addition and multiplication on the complex 
numbers define a field. Both $ \complex $ and $ \real $ are fields. Most 
theorems in linear algebra hold for fields, so all further statements will use 
the terminology $ \field $ to refer to either $ \complex $ or $ \real $.

\begin{defn}{$n$-tuples}
    Let $ n $ be a nonnegative integer. An \textbf{$n$-tuple} is an ordered list
    of $ n $ elements.
    \[
        (x_1,x_2,\dots,x_n)
    \]
    Two lists are equal if and only if they have the same length and the same
    elements in the same order.
\end{defn}
\end{document}

