\documentclass{article} 

% Document layout
\usepackage[
	margin=0.75in,
	headheight=17pt,
	includehead,
	includefoot,
	heightrounded,
]{geometry}

% Headers and footers
\usepackage{fancyhdr}
\fancyhf{}
\pagestyle{fancy}
\fancyhead[R]{Linear Algebra}
\fancyfoot[R]{Written by Jonathan Davidson}

% Colors
\usepackage[dvipsnames]{xcolor} 

% Tcolorbox
\usepackage{tcolorbox}
\tcbset{
	colback=NavyBlue!5!white,
	colframe=NavyBlue!75!black,
	fonttitle=\bfseries
}
\newtcbox{definition}[1]{
    title={#1},
}

% Math symbols and theorem environment 
\usepackage{amsmath,amsthm,amssymb}

% Multiple columns
\usepackage{multicol}

% Adjust enumerated lists
\usepackage[shortlabels]{enumitem} 

% Insert Images
\usepackage{graphicx}
\graphicspath{ {./images/} }

% Custom commands
\newcommand{\real}{\mathbb{R}} 
\newcommand{\complex}{\mathbb{C}} 
\newcommand{\set}[2]{\left\{#1\;\vert\;#2\right\}}

% DOCUMENT
\begin{document}
\section{Vector Spaces}
Linear algebra is the study of linear maps on finite dimensional vector spaces.
In linear algebra, complex numbers are necessary for classifying the basic types
of linear transformations in terms of their eigenvalues. Eigenvalues arise as
solutions to certain polynomial equations.

\begin{tcolorbox}[title=Complex Numbers]
    A \textbf{complex number} is an ordered pair $(a,b)$ of real numbers.
    Complex numbers are often written as $a+bi$. The set of all complex numbers
    is denoted $\complex$.
    \[
        \complex = \set{a+bi}{a,b\in\real}
    \]
\end{tcolorbox}

\end{document}

