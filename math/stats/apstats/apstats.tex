\documentclass{article}

% Document layout
\usepackage[
	margin=0.75in,
	headheight=17pt,
	includehead,
	includefoot,
	heightrounded,
]{geometry}

% Headers and footers
\usepackage{fancyhdr}
\fancyhf{}
\pagestyle{fancy}
\fancyhead[L]{AP Statistics}
\fancyhead[R]{Elite Prep Arcadia}
\fancyfoot[C]{\thepage}

% Tcolorbox
\usepackage{tcolorbox}
\tcbset{
	colback=gray!5!white,
	colframe=gray!75!black,
	fonttitle=\bfseries
}
\newtcolorbox{definition}{
    title=Definition,
}

% Utility packages
\usepackage{amsmath,amsthm,amssymb}
\usepackage{multicol}

% Insert Images
\usepackage{graphicx}
\graphicspath{ {./images/} }

% Exercise environment
\newtheorem*{remark}{Remark}
\theoremstyle{definition}
\newtheorem{ex}{Exercise}

% Custom commands

\begin{document}

\section{Data}
\begin{definition}
    \textbf{Categorical Variables} represent data that can be divided into different
    groups.
\end{definition}
\noindent Common examples of categorical data include ethnicity, income level,
education, age group, and gender. Categories are described by words or letters.
Categorical data is much harder to analyze mathematically than numerical data.
\begin{definition}
    \textbf{Quantitative Variables} represents numerical data from population
    measurement. Quantitative data can be either \textbf{discrete} or
    \textbf{continuous}.
\end{definition}
\noindent Common examples of quantitative data include height, weight, income, age, and
cost. If data set can only take on specific values (e.g. integer values), then
the data set is discrete. If the data is not restricted to specific values, then
the data set is continuous.

\section{Sampling}
Gathering the data on an entire population may be virtually impossible due to
limitations in time and cost. Instead, a \textbf{sample} of the population will
be studied. The sample must be representative of the population being sampled in
order for any statistical inference to be made. Random sampling is used to to
create samples that minimize any bias from the sampling process. 
\begin{definition}
    A \textbf{simple random sample (SRS)} is a sampling method in which each
    member of a population has an equal chance of being selected.
\end{definition}
\noindent In practice, it may be difficult to achieve a random sample. Care must be taken
to eliminate biases that arise in a sampling process. Many data selection
methods are not truly random. Random number generators can be powerful tools to
add randomization to the sampling process. Two other common types of sampling methods are 
stratified sampling and cluster sampling. Introductory statistics assumes a simple random 
sampling process.
    
\section{Histograms}

\section{Percentiles}

\section{Box Plots}

\section{Statistical Mean}
\begin{definition}
    The \textbf{mean (arithmetic mean)} is one of the most common measures of 
    center of a data set in statistics. The mean of a data set representing an 
    entire population is called a \textbf{population mean} and is denoted $\mu$. 
\end{definition}
\noindent The mean can be calculated by the formula 
\[
    \mu = \frac{1}{N}\sum x_i  
\]
where the sum is taken over the entire data set and $N$ represents the
population size. The mean of a data set representing a sample from a population 
is called a \textbf{sample mean} and is denoted $\bar{x}$. The formuala for 
sample mean is identical to the formula for population mean 
\[
    \bar{x} = \frac{1}{n}\sum x_i
\]
except that the sum is taken over the sample data set and $n$ represents the
sample size. For datasets described by histograms, the population mean formula
takes the form
\[
    \mu = \sum (RF_i) x_i
\]
where $RF_i$ is the relative frequency of the data point $x_i$.

\section{Skewness}










\end{document}

