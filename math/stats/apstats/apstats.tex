\documentclass{article}

% Document layout
\usepackage[
	margin=0.75in,
	headheight=17pt,
	includehead,
	includefoot,
	heightrounded,
]{geometry}

% Headers and footers
\usepackage{fancyhdr}
\fancyhf{}
\pagestyle{fancy}
\fancyhead[L]{AP Statistics}
\fancyhead[R]{Elite Prep Arcadia}
\fancyfoot[C]{\thepage}

% Tcolorbox
\usepackage{tcolorbox}
\tcbset{
	colback=gray!5!white,
	colframe=gray!75!black,
	fonttitle=\bfseries
}
\newtcolorbox{definition}{
    title=Definition,
}

% Utility packages
\usepackage{amsmath,amsthm,amssymb}
\usepackage{multicol}

% Insert Images
\usepackage{graphicx}
\graphicspath{ {./images/} }

% Exercise environment
\newtheorem*{remark}{Remark}
\theoremstyle{definition}
\newtheorem{ex}{Exercise}

% Custom commands

\begin{document}

\section{The Mean}
\begin{definition}
    The \textbf{mean (arithmetic mean)} is one of the most common measures of 
    center of a data set in statistics. The mean of a data set representing an 
    entire population is called a \textbf{population mean} and is denoted $\mu$. 
\end{definition}
\noindent The mean can be calculated by the formula 
\[
    \mu = \frac{1}{N}\sum x_i  
\]
where the sum is taken over the entire data set and $N$ represents the
population size. The mean of a data set representing a sample from a population 
is called a \textbf{sample mean} and is denoted $\bar{x}$. The formuala for 
sample mean is identical to the formula for population mean 
\[
    \bar{x} = \frac{1}{n}\sum x_i
\]
except that the sum is taken over the sample data set and $n$ represents the
sample size.

\end{document}

