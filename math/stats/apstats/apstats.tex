\documentclass{article}

% Document layout
\usepackage[
	margin=0.75in,
	headheight=17pt,
	includehead,
	includefoot,
	heightrounded,
]{geometry}

% Headers and footers
\usepackage{fancyhdr}
\fancyhf{}
\pagestyle{fancy}
\fancyhead[L]{AP Statistics}
\fancyhead[R]{Elite Prep Arcadia}
\fancyfoot[C]{\thepage}

% Tcolorbox
\usepackage{tcolorbox}
\tcbset{
	colback=cyan!5!white,
	colframe=cyan!75!black,
	fonttitle=\bfseries
}
\newtcolorbox{definition}{
    title=Definition,
}

% Utility packages
\usepackage{amsmath,amsthm,amssymb}
\usepackage{multicol}

% Insert Images
\usepackage{graphicx}
\graphicspath{ {./images/} }

% Exercise environment
\newtheorem*{remark}{Remark}
\theoremstyle{definition}
\newtheorem{ex}{Exercise}

% Custom commands
\newcommand{\secend}[0]{\noindent\rule[0.5ex]{\linewidth}{1pt}} 

\begin{document}

\section{Data}
\begin{definition}
    \textbf{Categorical Variables} represent data that can be divided into different
    groups.
\end{definition}

\noindent Common examples of categorical data include ethnicity, income level,
education, age group, and gender. Categories are described by words or letters.
Categorical data is much harder to analyze mathematically than numerical data.
\begin{definition}
    \textbf{Quantitative Variables} represents numerical data from population
    measurement. Quantitative data can be either \textbf{discrete} or
    \textbf{continuous}.
\end{definition}

\noindent Common examples of quantitative data include height, weight, income, age, and
cost. If data set can only take on specific values (e.g. integer values), then
the data set is discrete. If the data is not restricted to specific values, then
the data set is continuous. 

\secend

\section{Sampling}
Gathering the data on an entire population may be virtually impossible due to
limitations in time and cost. Instead, a \textbf{sample} of the population will
be studied. The sample must be representative of the population being sampled in
order for any statistical inference to be made. Random sampling is used to to
create samples that minimize any bias from the sampling process. 

\begin{definition}
    A \textbf{simple random sample (SRS)} is a sampling method in which each
    member of a population has an equal chance of being selected.
\end{definition}

\noindent In practice, it may be difficult to achieve a random sample. Care must be taken
to eliminate biases that arise in a sampling process. Many data selection
methods are not truly random. Random number generators can be powerful tools to
add randomization to the sampling process. Two other common types of sampling methods are 
stratified sampling and cluster sampling. Introductory statistics assumes a simple random 
sampling process.
    
\secend

\section{Histograms}
\begin{definition}
    A \textbf{Histogram} is a bar chart representing how many data points are
    present in each specified interval.
\end{definition}

\noindent To construct a histogram from a data set:
\begin{enumerate}
    \item Determine number of bins.
    \item Find the bin width and determine the bin intervals.
    \item Count the number of data points per bin.
    \item Plot data point counts vs bins on a bar graph.
\end{enumerate}

\secend

\section{Percentiles}
\begin{definition}
    \textbf{Percentiles} measure the location of data relative to the entire
    data set. The $k$th percentile is a score below which $k$ percent of the
    data falls below.
    \tcblower
    The 25th percentiles is called the \textbf{first quartile} $Q_1$. The 50th
    percentile is called the \textbf{median} or \textbf{second quartile} $Q_2$.
    The 75th percentile is called the \textbf{third quartile} $Q_3$.
\end{definition}

\begin{figure}[h]
    \centering
    \includegraphics[width=0.8\textwidth]{boxplot.png}
\end{figure}

\begin{definition}
    A \textbf{box plot} is a diagram representing five values: the minimum,
    maximum, and the three quartiles.
    \tcblower
    To calculate a box plot on a TI-84 calculator:
    \begin{enumerate}
        \item STAT EDIT. Fill a list with values. 
        \item Press STAT and arrow to CALC
        \item 1-VarStats. Choose list with values entered. ENTER 
    \end{enumerate}
\end{definition}

\secend

\section{Statistical Mean}
\begin{definition}
    The \textbf{mean (arithmetic mean)} is one of the most common measures of 
    center of a data set in statistics. The mean of a data set representing an 
    entire population is called a \textbf{population mean} and is denoted $\mu$. 
\end{definition}

\noindent The mean can be calculated by the formula 
\[
    \mu = \frac{1}{N}\sum x
\]
where the sum is taken over the entire data set and $N$ represents the
population size. The mean of a data set representing a sample from a population 
is called a \textbf{sample mean} and is denoted $\bar{x}$. The formuala for 
sample mean is identical to the formula for population mean 
\[
    \bar{x} = \frac{1}{n}\sum x
\]
except that the sum is taken over the sample data set and $n$ represents the
sample size. For datasets described by frequency tables or histograms, 
\[
    \mu = \frac{\sum f m}{\sum f}
\]
where $f$ is the frequency of the interval and $m$ is the midpoint of the
interval.

\secend

\section{Standard Deviation}
\begin{definition}
    The \textbf{standard deviation} provides a measure of the overall variation
    in a data set. The standard deviation can be used to determine whether a
    data value is close to or far away from the mean.
\end{definition}

\noindent The \textbf{population standard deviation} $\sigma$ is computed by the
formula
\[
    \sigma = \sqrt{\frac{\sum(x-\mu)^2}{N}}
\]
where $\mu$ is the population mean and $N$ is the population size. The
\textbf{sample standard deviation} $s$ has a similar formula 
\[
    s = \sqrt{\frac{\sum(x-\bar{x})^2}{n-1}}
\]
where $\bar{x}$ is the sample mean and $n$ is the sample size. Observe that the
denominator is $n-1$ not $n$. 







\end{document}

