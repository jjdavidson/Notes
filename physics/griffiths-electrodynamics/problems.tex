\documentclass{article}

% Document Layout
\usepackage[
    margin=1in,
    headheight=17pt,
    includehead,
    includefoot,
    heightrounded,
]{geometry}
\usepackage{fancyhdr}

% Graphics
\usepackage{graphicx}
\graphicspath{ {./images/} }

% Theorem Environments
\usepackage{amsmath,amsfonts,amssymb,amsthm}
\theoremstyle{definition}
\newtheorem*{definition}{Definition}
\newtheorem{problem}{Problem}[section]
\theoremstyle{remark}
\newtheorem*{remark}{Remark}

% New commands
\newcommand{\pline}{\par\noindent\rule{\textwidth}{0.5pt}}
\newcommand{\abs}[1]{\left|#1\right|}
\newcommand{\components}[1]{\left\langle #1 \right\rangle}

\renewcommand{\vec}{\mathbf}
\newcommand{\veca}{\vec{A}}
\newcommand{\vecb}{\vec{B}}
\newcommand{\vecc}{\vec{C}}
\newcommand{\vecd}{\vec{D}}

\newcommand{\unitnorm}{\hat{\vec{n}}}
\newcommand{\xhat}{\hat{\vec{x}}}
\newcommand{\yhat}{\hat{\vec{y}}}
\newcommand{\zhat}{\hat{\vec{z}}}
\newcommand{\rhat}{\hat{\vec{r}}}

\newcommand{\cross}{\times}

\newcommand{\diff}[2]{\frac{\partial #1}{\partial #2}}
\newcommand{\grad}{\nabla}
\renewcommand{\div}{\nabla\cdot}
\newcommand{\curl}{\nabla/cross}

\begin{document}
% Styling
\renewcommand{\theenumi}{\alph{enumi}}
\pagestyle{fancy}
\fancyhead[L]{David J. Griffiths}
\fancyhead[R]{Introduction to Electrodynamics}
\fancyfoot[R]{Notes and Solutions by Jonathan Davidson}
\counterwithin{equation}{section}

\section{Vector Analysis}
\subsection{Vector Algebra}
\subsubsection{Vector Geometry}

\begin{definition}[Vector]
A \textbf{vector} is a quantity that has both \textit{magnitude} and \textit{direction}.
\end{definition}

\noindent A vector is represented geometrically as an arrow with the length of the arrow determined by the magnitude. Vectors will be identified with \textbf{math boldface} notation. The magnitude of a vector $\veca$ is written $\abs{\veca}$. A \textbf{scalar} is a quantity with magnitude but no direction. Vectors have magnitude and direction but no \textit{location}. Arrow diagrams can be slid at will as long as there is no change in direction or magnitude.

\begin{definition}[Vector Addition]
Place the tail of $\vecb$ at the head of $\veca$. The sum $\veca+\vecb$ is the vector from the tail of $\veca$ to the tail of $\vecb$.
\end{definition}

\noindent Addition is \textit{commutative} and \textit{associative}
\[\veca+\vecb = \vecb+\veca\]
\[\veca+(\vecb+\vecc) = (\veca+\vecb)+\vecc\]
Minus $\veca$ denoted $-\veca$ is a vector facing in the opposite direction. To subtract two vectors add the minus of the second vector
\[\veca-\vecb = \veca+(-\vecb)\]

\begin{definition}[Scalar Multiplication]
Multiplication of a vector by a positive scalar $a$ multiplies the magnitude but leaves the direction unchanged. If $a$ is negative, the direction will be reversed.
\end{definition}

\noindent Scalar multiplication is \textit{distributive}
\[a(\veca+\vecb) = a\veca+a\vecb\]

\begin{definition}[Dot Product]
The dot product of two vectors is defined by 
\begin{equation}
\veca\cdot\vecb\equiv\abs{\veca}\abs{\vecb}\cos{\theta}
\end{equation}
where $\theta$ is the angle between two vectors when placed tail-to-tail.  $\veca\cdot\vecb$ is a \textit{scalar}.
\end{definition}

\noindent The dot product is \textit{commutative} and \textit{distributive}
\[\veca\cdot\vecb = \vecb\cdot\veca\]
\begin{equation}
\veca\cdot(\vecb+\vecc) = \veca\cdot\vecb+\veca\cdot\vecc
\end{equation}
If $\veca$ and $\vecb$ are perpendicular, then $\veca\cdot\vecb = 0$. If the two vectors are parallel, then $\veca\cdot\vecb = \abs{\veca}\abs{\vecb}$. In particular,
\begin{equation}
\veca\cdot\veca = \abs{\veca}^{2}
\end{equation}

\begin{definition}[Cross Product]
The cross product of two vectors is defined by
\begin{equation}
\veca\cross\vecb\equiv\abs{\veca}\abs{\vecb}\sin{\theta}\ \unitnorm
\end{equation}
where $\unitnorm$ is a unit vector perpendicular to the plane of $\veca$ and $\vecb$. The direction is chosen according to the right hand rule. $\veca\cross\vecb$ is a vector.
\end{definition}

\noindent The cross product is \textit{distributive} and \textit{anticommutative}
\begin{equation}
\veca\cross(\vecb+\vecc) = \veca\cross\vecb+\veca\cross\vecc
\end{equation}
\begin{equation}
\vecb\cross\veca = -\veca\cross\vecb
\end{equation}
Geometrically, $\abs{\veca\cross\vecb}$ is the area of a parallelogram generated by $\veca$ and $\vecb$. If the vectors are parallel, their cross product is zero. In particular,
\[\veca\cross\veca = 0\]

\pline
\begin{problem}
Using the definitions in equations 1.1 and 1.4, and appropriate diagrams, show that the dot product and cross product are distributive.
\end{problem}

\includegraphics[width=0.5\textwidth]{distributive-dot.svg.png}
\includegraphics[width=0.4\textwidth]{distributive-cross.png}
\pline
\begin{problem}
Is the cross product associative?
\[(\veca\cross\vecb)\cross\vecc\overset{?}{=}\veca\cross(\vecb\cross\vecc)\]
\end{problem}
The cross product is not associative in general. Suppose $\veca$ and $\vecb$ are orthogonal unit vectors. Then, 
\[(\veca\cross\vecb)\cross\vecb = -\veca\]
whereas
\[\veca\cross(\vecb\cross\vecb) = \vec{0}\]
\pline

\subsubsection{Vector Components}

\begin{definition}[Components]
In rectangular coordinates, the \textbf{basis vectors} $\xhat$, $\yhat$, and $\zhat$ are unit vectors pointing in the directions of the $x$, $y$, and $z$ axes. An arbitrary vector $\veca$ can be expressed as a linear combination of these basis vectors
\[\veca = a_{1}\xhat+a_{2}\yhat+a_{3}\zhat = \components{a_1,a_2,a_3}\]
The coefficients $a_{1}$, $a_{2}$, and $a_{3}$ are called \textbf{components} of $\veca$.
\end{definition}

\noindent Vector operations can be restated in terms of components.
\begin{equation}
\veca+\vecb = (a_{1}+b_{1})\xhat+(a_{2}+b_{2})\yhat+(a_{3}+b_{3})\zhat
\end{equation}
\begin{equation}
\alpha\veca = (\alpha a_{1})\xhat+(\alpha a_{2})\yhat+(\alpha a_{3})\zhat
\end{equation}
\stepcounter{equation}
\begin{equation}
\veca\cdot\vecb=a_1b_1+a_2b_2+a_3b_3
\end{equation}
\addtocounter{equation}{2}
\begin{equation}
\veca\cross\vecb=(a_2b_3-a_3b_2)\xhat+(a_3b_1-a_1b_3)\yhat+(a_1b_2-a_2b_1)\zhat
\end{equation}
\begin{remark}
The cross product can be calculated as the determinant of a matrix whose first row is $[\xhat\quad\yhat\quad\zhat]$, whose second row is the components $[a_1 \quad a_2 \quad a_3]$, and whose third row is the components $[b_1 \quad b_2 \quad b_3]$.
\begin{equation}
\veca\cross\vecb = 
\begin{vmatrix}
\xhat & \yhat & \zhat \\
a_1 & a_2 & a_3 \\
b_1 & b_2 & b_3
\end{vmatrix}
\end{equation}
\end{remark}

\pline
\begin{problem}
Find the angle between the body diagonals of a cube
\end{problem}
Without loss of generality, consider the unit cube with vertices $(0,0,0)$, $(1,0,0)$, $(0,1,0)$, $(0,0,1)$, $(1,1,0)$, $(1,0,1)$, $(0,1,1)$, and $(1,1,1)$. The body diagonals are given by the vectors
\begin{align*}
\veca_1&=\components{1,1,1} \\
\veca_2&=\components{-1,1,1} \\
\veca_3&=\components{1,-1,1} \\
\veca_4&=\components{1,1,-1} 
\end{align*}
For, $i,j$ in the range $\{1,2,3,4\}$, 
\[\veca_i\cdot\veca_j = \pm1,\quad \abs{\veca_i} = \sqrt{3}\]
The angle between the vectors representing the diagonals satisfies
\[cos(\theta)=\frac{\veca_i\cdot\veca_j}{\abs{\veca_i}\abs{\veca_j}} = \pm\frac{1}{3}, \quad \theta \approx 70.53^{\circ}\]
\pline
\begin{problem}
Use the cross product to find the components of the unit vector $\unitnorm$ perpendicular to the plane passing through the points $(1,0,0)$, $(0,2,0)$, and $(0,0,3)$.
\end{problem}
$\veca=\components{-1,2,0}$ and $\vecb=\components{-1,0,3}$ are vectors inside the plane since they are displacement vectors from $(1,0,0)$ to $(0,2,0)$ and $(0,0,3)$ respectively. Since, 
\[\veca\cross\vecb=\components{6,3,2},\quad \abs{\components{6,3,2}} = 7\]
Therefore, \[\unitnorm=\components{\frac{6}{7},\frac{3}{7},\frac{2}{7}}\]
\pline

\subsubsection{Triple Products}

\begin{definition}[Scalar Triple Product]
Geometrically, $\abs{\veca\cdot(\vecb\cross\vecc)}$ represents the area of a parallelpiped generated by the $\veca$, $\vecb$, and $\vecc$.
\end{definition}
\begin{equation}
\veca\cdot(\vecb\cross\vecc)=\vecb\cdot(\vecc\cross\veca)=\vecc\cdot(\veca\cross\vecb)
\end{equation}
In component form,
\begin{equation}
\veca\cdot(\vecb\cross\vecc) = 
\begin{vmatrix}
a_1 & a_2 & a_3 \\
b_1 & b_2 & b_3 \\
c_1 & c_2 & c_3
\end{vmatrix}
\end{equation}

\begin{definition}[Vector Triple Product]
The vector triple product follows the \textbf{BAC-CAB} rule:
\begin{equation}
\veca\cross(\vecb\cross\vecc)=\vecb(\veca\cdot\vecc)-\vecc(\veca\cdot\vecb)
\end{equation}
\end{definition}

\noindent All higher order vector products can be simplified into an expression with no more than one cross product per term. For instance,
\begin{equation}
\begin{split}
(\veca\cross\vecb)\cdot(\vecc\cross\vecd) &= (\veca\cdot\vecc)(\vecb\cdot\vecd) - (\veca\cdot\vecd)(\vecb\cdot\vecc) \\
(\veca\cross(\vecb\cross(\vecc\cross\vecd))) &= \vecb(\veca\cdot(\vecc\cross\vecd)) - (\veca\cdot\vecb)(\vecc\cross\vecd)
\end{split}
\end{equation}

\pline
\begin{problem}
Prove the \textbf{BAC-CAB} rule by writing out both sides in component form.
\end{problem}
\begin{align*}
\veca\cross(\vecb\cross\vecc) = &\langle a_1,a_2,a_3 \rangle\cross\components{b_2c_3-b_3c_2,b_3c_1-b_1c_3,b_1c_2-b_2c_1} \\
= &\langle a_2b_1c_2-a_2b_2c_1+a_3b_1c_3-a_3b_3c_1, \\
&\; a_3b_2c_3-a_3b_3c_2+a_1b_2c_1-a_1b_1c_2, \\
&\; a_1b_3c_1-a_1b_1c_3+a_2b_3c_2-a_2b_2c_3\rangle \\
= &\langle a_2b_1c_2-a_2b_2c_1+a_3b_1c_3-a_3b_3c_1+a_1b_1c_1-a_1b_1c_1, \\
&\; a_3b_2c_3-a_3b_3c_2+a_1b_2c_1-a_1b_1c_2+a_2b_2c_2-a_2b_2c_2,\\
&\; a_1b_3c_1-a_1b_1c_3+a_2b_3c_2-a_2b_2c_3+a_3b_3c_3-a_3b_3c_3\rangle \\
= &\langle b_1(a_1c_1+a_2c_2+a_3c_3) -c_1(a_1b_1+a_2b_2+a_3b_3), \\
&\; b_2(a_1c_1+a_2c_2+a_3c_3) -c_2(a_1b_1+a_2b_2+a_3b_3), \\
&\; b_3(a_1c_1+a_2c_2+a_3c_3) -c_3(a_1b_1+a_2b_2+a_3b_3), \\
= & \vecb(\veca\cdot\vecc)-\vecc(\veca\cdot\vecb)
\end{align*}
\pline
\begin{problem}
Prove the Jacobi identity
\[\veca\cross(\vecb\cross\vecc) + \vecb\cross(\vecc\cross\veca) + \vecc\cross(\veca\cross\vecb) = 0\]
Under what conditions does $\veca\cross(\vecb\cross\vecc) = (\veca\cross\vecb)\cross\vecc$?
\end{problem}
Using equation (1.17) and commutativity of the dot product
\begin{align*}
&\veca\cross(\vecb\cross\vecc)+\vecb\cross(\vecc\cross\veca)+\vecc\cross(\veca\cross\vecb) \\
&\quad=\vecb(\veca\cdot\vecc)-\vecc(\veca\cdot\vecb) + \vecc(\vecb\cdot\veca)-\veca(\vecb\cdot\vecc) + \veca(\vecc\cdot\vecb)-\vecb(\vecc\cdot\veca) = 0
\end{align*}
Suppose that the vector triple product of $\veca$, $\vecb$, and $\vecc$ were associative. \[\veca\cross(\vecb\cross\vecc) = (\veca\cross\vecb)\cross\vecc\] Since the cross product is anticommutative, \[\veca\cross(\vecb\cross\vecc)+\vecc\cross(\veca\cross\vecb) = 0\] Combining this observation with the Jacobi identity yields $\vecb\cross(\vecc\cross\veca) = 0$. Hence, the cross product is commutative if and only if $\veca$ is parallel to $\vecc$ or $\vecb$ is perpendicular to both $\veca$ and $\vecc$.
\pline

\subsubsection{Position and Displacement}
\begin{definition}[Position]
The \textbf{position vector} is a vector pointing from the origin to its Cartesian coordinates.
\begin{equation}
\vec{r}\equiv x\xhat+y\yhat+z\zhat
\end{equation}
\end{definition}
with magnitude
\begin{equation}
r = \sqrt{x^2+y^2+z^2}
\end{equation}
Converting to a unit vector, the position vector points out radially
\begin{equation}
\rhat = \frac{\vec{r}}{r} = \frac{x\xhat+y\yhat+z\zhat}{\sqrt{x^2+y^2+z^2}}
\end{equation}

\begin{definition}[Infintesimal Displacement Vector]
The \textbf{Infintesimal Displacement Vector} is a vector pointing from a position $(x,y,z)$ to $(x+dx,y+dy,z+dz)$ denoted by
\begin{equation}
d\vec{r} = dx\xhat+dy\yhat+dz\zhat
\end{equation}
\end{definition}

\addtocounter{equation}{6}

\pline
\begin{problem}
Find the separation vector from source point $(2,8,7)$ to the field point $(4,6,8)$. Determine its magnitude, and construct the unit separation vector.
\end{problem}
The separation vector is given by $\components{4-2,6-8,8-7}=\components{2,-2,1}$. It has magnitude \[\sqrt{2^2+(-2)^2+1^2}=3\] The unit separation vector is $\components{\frac{2}{3},\frac{-2}{3},\frac{1}{3}}$.
\pline

\subsubsection{Transforming Vectors}
Vectors are not defined solely as objects that have \textit{magnitude} and \textit{direction}. Their components also transform in a specific way under a change of coordinates. Consider the rotation transformation

\noindent In the $yz$-coordinate system, \[\veca = \components{\abs{\veca}\cos\theta,\abs{\veca}\sin\theta} = \components{a_{y},a_{z}}\] In the $\bar{y}\bar{z}$-coordinate system \[\veca = \components{\abs{\veca}\cos\bar{\theta},\abs{\veca}\sin\bar{\theta}} = \components{\bar{a}_y,\bar{a}_z}\] The angles are related by $\bar{\theta} = \theta-\phi$. Applying the angle addition formulas for sine and cosine yields the transformation law
\begin{equation}
\begin{bmatrix}
\bar{a}_y \\ \bar{a}_z
\end{bmatrix}
=\begin{bmatrix}
\cos\phi & \sin\phi \\ -\sin\phi & cos\phi
\end{bmatrix}
\begin{bmatrix}
a_y \\ a_z
\end{bmatrix}
\end{equation}
For a rotation abount an arbitrary axis in three dimensions, the transformation law becomes
\begin{equation}
\begin{bmatrix}
\bar{a}_x \\ \bar{a}_y \\ \bar{a}_z
\end{bmatrix}
=\begin{bmatrix}
R_{11} & R_{12} & R_{13}\\R_{21} & R_{22} & R_{23}\\R_{31} & R_{32} & R_{33}
\end{bmatrix}
\begin{bmatrix}
a_x \\ a_y \\ a_z
\end{bmatrix}
\end{equation}
or more compactly
\begin{equation}
\bar{a}_i = \sum R_{ij}a_j
\end{equation}
\stepcounter{equation}
\pline
\begin{problem} 
\end{problem}
\begin{description}
	\item{(a)} Prove that the two-dimensional rotation matrix (1.29) preserves dot products.
\begin{align*}
\veca \cdot \vecb &= \bar{a}_y\bar{b}_y + \bar{a}_z\bar{b}_z \\
&= (cos(\phi)a_y+sin(\phi)a_z)(cos(\phi)b_y+sin(\phi)b_z) \\
&\quad +(-\sin(\phi)a_y+\cos(\phi)a_z)(-\sin(\phi)b_y+\cos(\phi)b_z) \\
&=\cos^2(\phi)a_yb_y+ \sin(\phi)\cos(\phi)a_zb_y+ \cos(\phi)\sin(\phi)a_yb_z+ \sin^2(\phi)a_zb_z \\
&\quad +\sin^2(\phi)a_yb_y- \sin(\phi)\cos(\phi)a_zb_y- \cos(\phi)\sin(\phi)a_yb_z+ \cos^2(\phi)a_zb_z \\
&= a_yb_y+a_zb_z
\end{align*}

	\item{(b)} What constraints must elements $(R_{ij})$ of the three-dimensional rotation matrix (1.30) satisfy in order to preserve the length of $\veca$?

If the magnitude is preserved then the squared magnitude will also be preserved. Consequently,
\begin{align*}
\abs{\veca}^2 &= \sum_{i=1}^3\bar{a}_i^2 \\
&= \sum_{i=1}^3\left(\sum_{j=1}^3R_{ij}a_j\right)\left(\sum_{k=1}^3R_{ik}a_k\right) \\
&= \sum_{i=1}^3\sum_{j,k} R_{ij}R_{ik}a_ja_k \\
&= \sum_{j,k}\sum_{i=1}^3 R_{ij}R_{ik}a_ja_k \\
&= a_1^2+a_2^2+a_3^2
\end{align*}
Therefore, \[\sum_{i=1}^3R_{ij}R_{ik} = \delta_{jk} \quad \forall j,k\]
\end{description}
\pline
\begin{problem}
Find the transformation $R$ that describes a clockwise rotation by $120^{\circ}$ about an axis from the origin through the point $(1,1,1)$. 
\end{problem}
This rotation sends $\xhat \to \zhat$, $\yhat \to \xhat$, and $\zhat \to \yhat$. Therefore, the rotation is represented by the matrix
\[\begin{bmatrix}0 & 1 & 0 \\ 0 & 0 & 1 \\ 1 & 0 & 0 \end{bmatrix}\]
\pline
\begin{problem}
\end{problem}
\begin{description}
	\item{(a)} How do the components of a vector transform under a \textbf{translation} of coordinates?
	
	Since a vector is not constrained to a specific location, the vector can be moved to be centered at the origin of the new coordinate system. Therefore, the components of the vector are invariant under translation, $\veca \to \veca$.
	
	\item{(b)} How do the components of a vector transform under an \textbf{inversion} of coordinates ($\xhat \to -\xhat$, $\yhat \to -\yhat$, $\zhat \to -\zhat$)?
	
	The components of a vector will change sign under an inversion of coordinates. Therefore, $\veca \to -\veca$.
	
	\item{(c)} How does the cross product of two vectors transform under an inversion?
	
	Since the cross product is bilinear, $\vecc = \veca\cross\vecb \to (-\veca)\cross(-\vecb) = \veca\cross\vecb = \vecc$. Therefore, the cross product is invariant under an inversion.
	
	\item{(d)} How does the scalar triple product of three vectors transform under inversions.
	
	Since the dot product and cross product are both bilinear, 
	\[\veca\cdot(\vecb\cross\vecc) = (-\veca)\cdot((-\vecb)\cross(-\vecc)) = -(\veca\cdot(\vecb\cross\vecc))\] Therefore, the scalar triple product changes sign under inversion.
\end{description}
\pline

\subsection{Differential Calculus}

\subsubsection{Gradient}
In single variable calculus, the derivative the function $f(x)$ tells how rapidly a $f$ varies when we change the argument $x$ by a tiny amount $dx$:
\begin{equation}
df=\left(\frac{df}{dx}\right)dx
\end{equation}
In the case of three variables, \textbf{partial derivatives} record how much a function $f(x,y,z)$ changes in each coordinate direction. The total change in $f(x,y,z)$ is summarized by the relation 
\begin{equation}
df = \diff{f}{x}dx+\diff{f}{y}dy+\diff{f}{z}dz
\end{equation}
This equation can be rewritten using a dot product as 
\begin{equation}
df = \left(\diff{f}{x}\xhat+\diff{f}{x}\xhat+\diff{f}{x}\xhat\right) \cdot (dx\xhat+dy\yhat+dz\zhat) = (\grad f)\cdot (d\vec{r})
\end{equation}
\begin{definition}[Gradient]
The vector quantity
\begin{equation}
\grad f = \diff{f}{x}\xhat+\diff{f}{x}\xhat+\diff{f}{x}\xhat
\end{equation}
is called the gradient of $f$.
\end{definition}

\noindent The gradient $\grad f$ points in the direction of maximum increase of the function. Moreover, the magnitude of the gradient $\abs{\grad f}$ gives the rate of increase along this maximal direction.

\begin{definition}[Stationary point]
A point $(x,y,z)$ where the gradient vanishes is called a stationary point. 
\end{definition}

\noindent A stationary point could be a local maximum, local minimum, or a saddle point. These are the analagous to the maximum, minimum, and inflection point encountered in single variable calculus.

\pline
\begin{problem}
Find the gradient of the following functions:
\end{problem}
\begin{description}
\item{(a)} $f(x,y,z) = x^2+y^3+z^4$. \[\grad f = \components{2x,3y^2,4z^3}\]
\item{(b)} $f(x,y,z) = x^2y^3z^4$. \[\grad f=\components{2xy^3z^4,3x^2y^2x^4,4x^2y^3z^3}\]
\item{(c)} $f(x,y,z) = e^{x}\sin(y)\ln(z)$ \[\grad f=\components{e^{x}\sin(y)\ln(z),e^{x}\cos(y)\ln(z),e^{x}\sin(y)z^{-1}}\]
\end{description}

\pline
\begin{problem}
The height of a certain hill is given by \[h(x,y) = 10(2xy-3x^2-4y^2-18x+28y+12)\]
\end{problem}
\begin{description}
\item{(a)} Where is the top of the hill located?

The top of the hill will be located at a stationary point of the function, 
\item{(b)} How high is the hill?
\item{(c)} How steep is the slope at the point $(1,1)$? In what direction is the slope steepest at that point?
\end{description}
\pline

\begin{problem}
Show that
\end{problem}
\begin{description}
\item{(a)} $\grad (r^2) = 2\vec{r}$
\item{(b)} $\grad (r^{-1}) = -r^{-2}\rhat$
\item{(c)} What is the general formula for $\grad (r^n)$?
\end{description}










\subsection{Divergence}

\subsection{Curl}

\subsection{Product Rules}

\subsection{Second Derivatives}






\end{document}





























